%!TEX root = main.tex
% declare the root file

The conclusion enables you to reinforce the main messages of the document.
A conclusion summarizes the report as a whole,
drawing inferences from the entire process about what has been found, or decided,
and the impact of those findings or decisions.

Even in a short report, it is useful to include a conclusion.
A conclusion demonstrates good organization.
When written well, it can help make the reader's task easier.
With a good conclusion, you can pull all the threads of the report details together and
relate them to the initial purpose for writing the report.
In other words, the conclusion should confirm for the reader that the report's purpose has been achieved.\footnote{Referenced from: \url{http://colelearning.net/who/module3/page40.html} (27/05/2017)}
