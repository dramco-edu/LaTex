%%%%%%%%%%%%%%%%%%%%%%%%%%%%%%%%%%%%%%%%%
% LaTeX Template
% Contributors:
% - Gilles Callebaut
% - Guus Leenders
% - Stijn Crul
%
% Original author:
% Linux and Unix Users Group at Virginia Tech Wiki
% (https://vtluug.org/wiki/Example_LaTeX_chem_lab_report)
%
% License:
% CC BY-NC-SA 3.0 (http://creativecommons.org/licenses/by-nc-sa/3.0/)
%
%%%%%%%%%%%%%%%%%%%%%%%%%%%%%%%%%%%%%%%%%

%----------------------------------------------------------------------------------------
%	PACKAGES AND DOCUMENT CONFIGURATIONS
%----------------------------------------------------------------------------------------

\documentclass[twoside,a4paper]{article}
\usepackage{geometry}

\usepackage{siunitx} % Provides the \SI{}{} and \si{} command for typesetting SI units
\usepackage{graphicx} % Required for the inclusion of images
\usepackage{natbib} % Required to change bibliography style to APA
\usepackage{amsmath} % Required for some math elements
\usepackage{lastpage} % know last page (used in fancy header)
\usepackage{fancyhdr} % Fancy Header
\usepackage[xindy, toc, numberedsection]{glossaries} % glossaries with xindy (recommended) for the indexing phase and show glossaries in TOC, and numberedsection to get a Setion number in the title
\usepackage{url} %The command is intended for email addresses, hypertext links, directories/paths, etc., which normally have no spaces, so by default the package ignores spaces in its argument.
\usepackage{minted} % it allows formatting and highlighting source code
% in addition, Pygments must be installed
% How to install on Windows:
% 1) install python (and add it to your PATH)
% 2) install pip (https://pip.pypa.io/en/stable/installing/)
% 3) install pygments (pip install Pygments)
% add pygments to your PATH, the command "pip show Pygments" shows where the lib is installed
% Of course, pdfLaTeX (or whatever engine/format you use) still has to be called with the -shell-escape option.



\loadglsentries{glossaries.tex}
\makeglossaries % generate the glossary
% Any links in resulting glossary will not be "clickable" unless you load the glossaries package after the hyperref package.

\setlength\parindent{0pt} % Removes all indentation from paragraphs

\renewcommand{\labelenumi}{\alph{enumi}.} % Make numbering in the enumerate environment by letter rather than number (e.g. section 6)

\renewcommand{\arraystretch}{1.2} % Increasing the array stretch factor using \renewcommand{\arraystretch}{<factor>} where <factor> is a numeric value

%\usepackage{times} % Uncomment to use the Times New Roman font

%----------------------------------------------------------------------------------------
%	DOCUMENT INFORMATION
%----------------------------------------------------------------------------------------
\newcommand{\maintitle}{Information Transfer Model}
\newcommand{\course}{Multimedia Networks}
\newcommand{\coursenumber}{B-KUL-JPI125}
\newcommand{\class}{4ELICT}

%%%%%%%%%%%%%%%%%%%%%%%%%%%%%% HEADER %%%%%%%%%%%%%%%%%%%%%%%%%%%%%%
\pagestyle{fancy}
\fancyhf{}
\fancyhead[LE,RO]{\course}
\fancyhead[RE,LO]{\maintitle}
% if working with chapters
% \fancyfoot[CE,CO]{\leftmark}
\fancyfoot[LE,RO]{\thepage\ of \pageref{LastPage}}
%%%%%%%%%%%%%%%%%%%%%%%%%%%%%% HEADER %%%%%%%%%%%%%%%%%%%%%%%%%%%%%%


%%------------------------------ TITLE PAGE -----------------------------
\title{\maintitle \\ \course \\{\small\ (\coursenumber)}} % Title

\author{John \textsc{Smith}} % Author name

\date{\today} % Date for the report

\begin{document}\sloppy % sloppy is used to enforce that lines are in hbox
\newgeometry{hmarginratio=1:1}    %% make layout symmetric
\pagenumbering{gobble}% Remove page numbers (and reset to 1)
\begin{titlepage}
\maketitle % Insert the title, author and date

\vfill
\begin{center}
\includegraphics[width=0.13\textwidth]{logo_kuleuven.png} %
\end{center}
%each \vfill will expand vertically the same amount until the entire page is filled
\vfill
\vfill
\vfill

\begin{center}
\begin{tabular}{l r}
Session Date: & January 1, 2017 \\ % Date the experiment was performed
\\
Partners: & James Smith \\ % Partner names
& Mary Smith \\
Class: & \class \\
\\
Instructor: & Gilles Callebaut % Instructor/supervisor
\end{tabular}
\end{center}
\vfill
\vfill
\end{titlepage}
\clearpage
\newpage\null\thispagestyle{empty}\newpage % blank page after title page

%%------------------------------ TITLE PAGE -----------------------------


\restoregeometry%              %% restore the layout
\pagenumbering{arabic}% Arabic page numbers (and reset to 1)

\tableofcontents

\listoflistings%
\clearpage



%----------------------------------------------------------------------------------------
%	SECTION 1
%----------------------------------------------------------------------------------------

\section{Introduction}
Every report starts with an introduction. In this section the reader is introduced to why an what is being discussed in this report.
Furthermore, this section should also include a brief overview of the outline of the report.

\subsection{Objectives}
The aims and objectives of the report should be explained in detail, without presenting any solutions.
For instance, the objectives for writing report are:\footnote{Referenced from: \url{http://www2.le.ac.uk/offices/ld/resources/writing/writing-resources/reports} (27/05/2017)}
\begin{itemize}
  \item to find out what you have learned from your reading, research or experience;
  \item to give you experience of an important skill that is widely used in the work place
\end{itemize}

%----------------------------------------------------------------------------------------
%	SECTION  AC + GLOS
%----------------------------------------------------------------------------------------
\glsaddall % add all gls/aronyms, used or not
\printglossary[title={Definitions and Acronyms}]% print the glossaries
It is useful to provide a list of technical or new terms with a brief, clear description of each term. You can also include in this section explanations of the acronyms, abbreviations or standard units used in your report.


%----------------------------------------------------------------------------------------
%	SECTION 2
%----------------------------------------------------------------------------------------

\section{Experimental Data}

\begin{listing}[ht]
  \centering
\begin{minted}[linenos,tabsize=2,breaklines, fontsize=\footnotesize]{python}
import numpy as np

def incmatrix(genl1,genl2):
    m = len(genl1)
    n = len(genl2)
    M = None #to become the incidence matrix
    VT = np.zeros((n*m,1), int)  #dummy variable

    #compute the bitwise xor matrix
    M1 = bitxormatrix(genl1)
    M2 = np.triu(bitxormatrix(genl2),1)

    for i in range(m-1):
        for j in range(i+1, m):
            [r,c] = np.where(M2 == M1[i,j])
            for k in range(len(r)):
                VT[(i)*n + r[k]] = 1;
                VT[(i)*n + c[k]] = 1;
                VT[(j)*n + r[k]] = 1;
                VT[(j)*n + c[k]] = 1;

                if M is None:
                    M = np.copy(VT)
                else:
                    M = np.concatenate((M, VT), 1)

                VT = np.zeros((n*m,1), int)

    return M
\end{minted}
\caption{Example Python code}
\label{listing:example}
\end{listing}

\begin{tabular}{ll}
Mass of empty crucible & \SI{7.28}{\gram}\\
Mass of crucible and magnesium before heating & \SI{8.59}{\gram}\\
Mass of crucible and magnesium oxide after heating & \SI{9.46}{\gram}\\
Balance used & \#4\\
Magnesium from sample bottle & \#1
\end{tabular}

%----------------------------------------------------------------------------------------
%	SECTION 3
%----------------------------------------------------------------------------------------

\section{Sample Calculation}

\begin{tabular}{ll}
Mass of magnesium metal & = \SI{8.59}{\gram} - \SI{7.28}{\gram}\\
& = \SI{1.31}{\gram}\\
Mass of magnesium oxide & = \SI{9.46}{\gram} - \SI{7.28}{\gram}\\
& = \SI{2.18}{\gram}\\
Mass of oxygen & = \SI{2.18}{\gram} - \SI{1.31}{\gram}\\
& = \SI{0.87}{\gram}
\end{tabular}

Because of this reaction, the required ratio is the atomic weight of magnesium: \SI{16.00}{\gram} of oxygen as experimental mass of Mg: experimental mass of oxygen or $\frac{x}{1.31}=\frac{16}{0.87}$.

%----------------------------------------------------------------------------------------
%	SECTION 4
%----------------------------------------------------------------------------------------

\section{Results and Conclusions}
Resultaten altijd eens valideren.

The atomic weight of magnesium is concluded to be \SI{24}{\gram\per\mol}, as determined by the stoichiometry of its chemical combination with oxygen. This result is in agreement with the accepted value.

\begin{figure}[h]
\begin{center}
\includegraphics[width=0.65\textwidth]{placeholder} % Include the image placeholder.png
\caption{Figure caption.}
\end{center}
\end{figure}

%----------------------------------------------------------------------------------------
%	SECTION 5
%----------------------------------------------------------------------------------------

\section{Discussion of Experimental Uncertainty}

The accepted value (periodic table) is \SI{24.3}{\gram\per\mole} \cite{dds:specification}.
The percentage discrepancy between the accepted value and the result obtained here is 1.3\%.
Because only a single measurement was made,
it is not possible to calculate an estimated standard deviation.

The most obvious source of experimental uncertainty is the limited precision of the balance.
Other potential sources of experimental uncertainty are: the reaction might not be complete;
if not enough time was allowed for total oxidation,
less than complete oxidation of the magnesium might have, in part,
reacted with nitrogen in the air (incorrect reaction);
the magnesium oxide might have absorbed water from the air,
and thus weigh ``too much.''
Because the result obtained is close to the accepted value it is possible that some of these experimental uncertainties have fortuitously cancelled one another.

%----------------------------------------------------------------------------------------
%	SECTION 6
%----------------------------------------------------------------------------------------

\section{Answers to Definitions}

\begin{enumerate}
\begin{item}
The \emph{atomic weight of an element} is the relative weight of one of its atoms compared to C-12 with a weight of 12.0000000$\ldots$, hydrogen with a weight of 1.008, to oxygen with a weight of 16.00. Atomic weight is also the average weight of all the atoms of that element as they occur in nature.
\end{item}
\begin{item}
The \emph{units of atomic weight} are two-fold, with an identical numerical value. They are g/mole of atoms (or just g/mol) or amu/atom.
\end{item}
\begin{item}
\emph{Percentage discrepancy} between an accepted (literature) value and an experimental value is
\begin{equation*}
\frac{\mathrm{experimental\;result} - \mathrm{accepted\;result}}{\mathrm{accepted\;result}}
\end{equation*}
\end{item}
\end{enumerate}

%----------------------------------------------------------------------------------------
%	CONCLUSION
%----------------------------------------------------------------------------------------
\section{Conclusion}
%!TEX root = main.tex
% declare the root file

The conclusion enables you to reinforce the main messages of the document.
A conclusion summarizes the report as a whole,
drawing inferences from the entire process about what has been found, or decided,
and the impact of those findings or decisions.

Even in a short report, it is useful to include a conclusion.
A conclusion demonstrates good organization.
When written well, it can help make the reader's task easier.
With a good conclusion, you can pull all the threads of the report details together and
relate them to the initial purpose for writing the report.
In other words, the conclusion should confirm for the reader that the report's purpose has been achieved.\footnote{Referenced from: \url{http://colelearning.net/who/module3/page40.html} (27/05/2017)}


%----------------------------------------------------------------------------------------
%	BIBLIOGRAPHY
%----------------------------------------------------------------------------------------
\clearpage
\bibliographystyle{apalike}

\bibliography{bib}

%----------------------------------------------------------------------------------------

\end{document}
